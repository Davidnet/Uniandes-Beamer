\documentclass{beamer}
\usetheme{AnnArbor}
\usepackage[spanish]{babel}
\usepackage{lmodern}
\usepackage[T1]{fontenc}
\usepackage[utf8]{inputenc}
\usepackage{amsmath}
\usepackage{amsfonts}
\usepackage{amssymb}
\usepackage{graphicx}
\usepackage{hyperref}
\author{Prof. Sebastian Saaibi \& David Cardozo\inst{1}}
% - Give the names in the same order as the appear in the paper.
% - Use the \inst{?} command only if the authors have different
%   affiliation.
\title{Temas Avanzados de  \LaTeXe }
\subtitle{Ecuaciones avanzadas, Bib\LaTeX, Herramientas de lineas de comando, Una hoja de vida} % A subtitle is optional and this may be deleted
%\logo{\includegraphics[height=0.8cm]{universidaddelosandes.png}\vspace{220pt}} 
\logo{\includegraphics[height=0.8cm]{universidaddelosandesciencias.png}}
%\logo{\includegraphics[height=0.8cm]{universidaddelosandescolombia.png}
\institute[Universidad de los Andes]
{
	\inst{1}%
	Física   \\
	Lectura $2$ Herramientas Computacionales \\
	Universidad de los Andes
}
\date{\today} % - Either use conference name or its abbreviation.
\subject{PDF Information} % This is only inserted into the PDF information catalog. Can be left out. 
%\setbeamercovered{transparent}
%\setbeamertemplate{navigation symbols}{}

\begin{document}
	\maketitle
	
	\begin{frame}
		\frametitle{Repaso de la anterior clase de herramientas}
		\begin{block}
			\centering
			{En la ultima clase aprendimos a crear documentos básicos para la presentación de tareas y reportes de laboratorios  }
		\end{block}	
		Para ello, debe crear un nuevo documento en \LaTeX que contenga las siguientes ecuaciones.
		
	\end{frame}
	\begin{frame}
		\frametitle{Actividad de Transcribir Ecuaciones}
		\begin{block}
			\centering
			{Transcribir en un documento de clase \emph{Report}}
		\end{block}
		\begin{itemize}
			\item $ A = \left\lbrace f(x) | \iiint \vec{\Delta} f d\vec{s} = 0 \right\rbrace  $
			\item 
			\[ \dfrac{\left| d(x,y) - d(z,x) \right| }{\frac{1}{2}} < \epsilon \quad \forall \epsilon < 0 \]
			\item $ L^2 = \left\lbrace f \in  \mathbb{R}^{\mathbb{R}} | \iint_{\infty}^{\infty}  f(x)dxdy \textrm{ es integrable y normalizable} \right\rbrace $			
		\end{itemize}
		\[ \begin{pmatrix}
		\dfrac{d}{dx_1} & \dfrac{d}{dx_2} & \ldots & \dfrac{d}{dx_n} \\
		\dfrac{d}{dx_1} \left( \dfrac{d}{dx_1} \right) & \dfrac{d}{dx_2} \left( \dfrac{d}{dx_2} \right)  & \ldots & \dfrac{d}{dx_n} \left( \dfrac{d}{dx_n} \right) \\
		\vdots & \vdots & \vdots \\
		\dfrac{d}{dx_1} \ldots \left( \dfrac{d}{dx_1} \right) & \dfrac{d}{dx_2} \ldots \left( \dfrac{d}{dx_2} \right)  & \ldots & \dfrac{d}{dx_n} \ldots \left( \dfrac{d}{dx_n} \right) 
		\end{pmatrix} \]
	\end{frame}
	
	
	
\end{document}